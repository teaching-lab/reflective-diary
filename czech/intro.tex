% ===== INTRODUCTION =====

\chapter*{Jak se z člověka stane dobrý učitel?}

\vspace*{1em}
Položme si namísto statické otázky dynamickou:

\vspace*{1em}
\textit{\large \enquote{Co mohu dělat pro to, abych se jako učitel zlepšoval?}}

\vspace*{1em}
Ukazuje se, že pro zlepšení jsou nutné (a nejspíš i dostačující) tři věci\punct{.}\footnotemark
\footnotetext{Jsou i další věci, které mohou pomoci (například expertní zpětná vazba), ale ty nejsou dobře škálovatelné.}
\begin{enumerate}
\item \textbf{Pravidelně učit.}\\Potřebuješ mít vlastní zkušenost, nejlépe často a pravidelně opakovanou.
\item \textbf{Reflektovat svoje učení.}\\Všímej si a přemýšlej nad tím, co fungovalo dobře a co příště zkusíš jinak.
\item \textbf{Pozorovat, jak učí ostatní.}\\Přemýšlej o tom, co dělají ostatní učitelé, co jim funguje, co ne a co z toho můžeš použít.
\end{enumerate}

Tento deníček ti pomůže reflektovat způsob, jakým učíš. Nalezneš v~něm informace o tom, čeho si všímat, jaké otázky si klást a nad čím přemýšlet. Není to však učebnice dobrého učitele -- je to pouze průvodce na cestě, po které už jdeš.

Vhodným doplňkem k vlastní reflexi výuky jsou diskuze s ostatními učiteli o tom, jak učí. Pokud se chceš setkávat s dalšími cvičícími pravidelně a navíc se dozvědět více o tom, co funguje a co ne, zapiš si předmět \textit{Teaching Lab} na Fakultě informatiky Masarykovy univerzity.

\newpage
\section*{Jak používat reflektivní deník?}

Především je důležité používat ho pravidelně.

Zapiš si svoje myšlenky při přípravě hodiny i po odvedené výuce. Máš nachystaných čtrnáct dvoustránek (jednu na každý týden semestru), přičemž každá nabízí pár otázek. Deníček je malý -- to mimo jiné i proto, aby ti psaní nezabralo moc času a tvé poznámky byly krátké.

Za stránkami pro jednotlivé týdny najdeš ještě rubriku učitelských schopností, seznam indikátorů, které je možné na hodině sledovat, seznam některých užitečných nástrojů k výuce a na závěr prostor na své vlastní poznámky.

Někde záměrně používáme anglické termíny (jako například \textit{tracking}), protože vhodný český překlad buď neexistuje, nebo má konotace, které nechceme.

\section*{Proč používat reflektivní deník?}

Pravidelná práce s deníkem plní vícero cílů:
\begin{enumerate}[topsep=0pt]
\item Pomáhá ti zamyslet se nad svou hodinou (reflexe).
\item Nabízí prostor pro snadné psaní si poznámek pro příště.
\item Pomáhá ti uvědomit si, co všechno patří k tvojí roli učitele.
\item Umožňuje ti sledovat svůj vlastní pokrok.
\end{enumerate}

\section*{Máš nápady na zlepšení?}

Budeme velice rádi, když nám sdělíš svoje návrhy na zlepšení nebo postřehy z používaní deníku. Napiš nám email na adresu \textit{teachinglab@fi.muni.cz}.
