% ===== REFLECTION ON INDIVIDUAL WEEKS =====

% Margin note
\newcommand{\marginsection}[3]{\marginpar{\raisebox{#2\height}{
\begin{turn}{#1}
\bfseries \color{gray} \large
#3\end{turn}}}}

% Timeline for planning the seminar structure
\newcommand{\timeline}{
\vspace*{2.1cm}
\color{gray}
\begin{chronology}[15]{0}{120}{\textwidth}
\end{chronology}
\vspace*{-0.2cm}
\color{black}
}

% Smileys for evaluating teacher's own feelings
\newcommand{\smileys}{
What is my lecture satisfatcion?
\raisebox{-0.3em}{\,\,\Changey[2]{-2}\,\,\Changey[2]{-1}\,\,\Changey[2]{0}\,\,\Changey[2]{1}\,\,\Changey[2]{2}}
}

% Summarizing 2 good and 2 bad points of the week
\newcommand{\goodbadpoints}{
What worked well?
\begin{enumerate}
\item
\item
\end{enumerate}

What could have been better?
\begin{enumerate}
\item
\item
\end{enumerate}
}

% Common week
\newcommand{\commonweek}{
\chapter{}
\vspace*{-2em}

\marginsection{90}{-1}{before teaching}

Planning:\hspace{1cm} hours

What is the goal of the lecture? Why am I teaching this?
\vspace*{1cm}

How do I know I reached my goal?

\hspace*{-1cm}
\rule{\rulelength}{0.4pt}

\marginsection{90}{-3}{after teaching}

\smileys

\goodbadpoints

Your own questions:

\newpage
\marginsection{-90}{1}{notes and comments}
}

\newgeometry{top=0.75cm, bottom=1.5cm, inner=0.75cm, outer=1.2cm,
			heightrounded, marginparwidth=0.5cm, marginparsep=0.3cm}

% ======= WEEK 1 =======

\chapter{}
\vspace*{-2em}

\marginsection{90}{-1.2}{fill in before teaching}

Time spent planning:\hspace{1cm} hours\\
\note{Preparing slides, excercises, tasks, ...}

What is the structure of the lecture?\\
\note{Outline 2--6 blocks on the timeline below.}

\timeline

What precedents do I want to set?\\
\note{Names, (in)formality, asking questions, starting on time, ...}
\vspace*{1cm}

\hspace*{-1cm}
\rule{\rulelength}{0.4pt}

\marginsection{90}{-1.3}{fill in after teaching}

\smileys
\note{What feeling do I have? What feeling do students seem to have?}

\goodbadpoints

\newpage
\marginsection{-90}{1}{notes and comments}
\vspace*{-2em}
\note{What (un)wanted precedents arose during the lecture?\\
How do I reinforce the wanted ones and supress the unwanted?\\
What was the prevailing climate/mood during the lecture?\\
Do the students understand the course structure?\\
Do they know what they are expected (not) to do?\\
(More ideas to ponder can be found on page \pageref{indicators}.)}

% ======= WEEK 2 =======

\chapter{}
\vspace*{-2em}

\marginsection{90}{-1.2}{fill in before teaching}

Time spent planning:\hspace{1cm} hours\\
\note{Preparing slides, excercises, tasks, ...}

What is the structure of the lecture?\\
\note{Outline 2--6 blocks on the timeline below.}

\timeline

What precedents do I want to set?\\
\note{Names, (in)formality, asking questions, starting on time, ...}
\vspace*{1cm}

\hspace*{-1cm}
\rule{\rulelength}{0.4pt}

\marginsection{90}{-1.3}{fill in after teaching}

\smileys
\note{What feeling do I have? What feeling do students seem to have?}

\goodbadpoints

\newpage
\marginsection{-90}{1}{notes and comments}
\vspace*{-2em}
\note{On what aspects of teaching shall I concentrate when observing my colleagues?\\
What aspect should colleagues visiting my lecture concentrate on?\\
(More ideas to ponder can be found on page \pageref{indicators}.)}

% ======= WEEK 3 =======

\chapter{}
\vspace*{-2em}

\marginsection{90}{-1.1}{fill in before teaching}

Planning:\hspace{1cm} hours\\

What is the structure of the lecture?

\timeline

Jaké otázky na hodině položím skupině?
\vspace*{0.8cm}

\hspace*{-1cm}
\rule{\rulelength}{0.4pt}

\marginsection{90}{-1.5}{vyplnit po výuce}

\smileys

\goodbadpoints

Na které otázky skupina reagovala? Na které ne?

\newpage
\marginsection{-90}{1}{volné poznámky}
\vspace*{-2em}
\note{Kolik otázek do publika jsem položil(a)?\\
Kolik otázek položili studenti mně? Je to málo/dost/příliš?\\
(Další tipy na otázky najdeš na straně \pageref{indikatory}.)}

% ======= WEEK 4 =======

\chapter{}
\vspace*{-2em}

\marginsection{90}{-1}{vyplnit před výukou}

Příprava:\hspace{1cm} hodin(y)

Z jakých bloků se moje hodina skládá?

\timeline

\hspace*{-1cm}
\rule{\rulelength}{0.4pt}

\marginsection{90}{-1.6}{vyplnit po výuce}

\smileys

\goodbadpoints

\note{Sem napiš otázku, na kterou si chceš po výuce odpovědět.}\\
Vlastní otázka:

\newpage
\marginsection{-90}{1}{volné poznámky}
\vspace*{-2em}
\note{Jaké znalosti a dovednosti chci naučit? Popiš je ve formátu rubriky, tedy pojmenuj dovednost a popiš škálu (viz str.\ \pageref{rubrika}).\\
Jaký typ učitele chci být? Co bych měl(a) pro to dělat?\\
(Další tipy na otázky najdeš na straně \pageref{indikatory}.)}

% ======= WEEK 5 =======

\chapter{}
\vspace*{-2em}

\marginsection{90}{-1.2}{před výukou}
Příprava:\hspace{1cm} hodin(y)

Jaký je cíl této hodiny? Proč chci studenty učit tohle? \\
\note{Při formulování cílů ti pomůžou slovesa na straně \pageref{bloom}.}
\vspace*{1cm}

Jak poznám, že se cíle podařilo dosáhnout? \\
\note{Existuje test, kterým je možné zjistit, zda jsi cíle dosáhl(a)?}

\hspace*{-1cm}
\rule{\rulelength}{0.4pt}

\marginsection{90}{-2.5}{po výuce}

\smileys

\goodbadpoints

\note{Sem napiš otázku, na kterou si chceš po výuce odpovědět.}\\
Vlastní otázka:

\newpage
\marginsection{-90}{1}{volné poznámky}
\vspace*{-2em}
\note{Podařilo se mi naplnit cíl hodiny?\\
Co je třeba do budoucna na téhle hodině změnit?\\
Jak budu plánovat a strukturovat nejbližší hodinu?\\
(Další tipy na otázky najdeš na straně \pageref{indikatory}.)}

% ======= WEEK 6 =======

\chapter{}
\vspace*{-2em}

\marginsection{90}{-1.2}{před výukou}

Příprava:\hspace{1cm} hodin(y)

Jaký je cíl této hodiny? Proč chci studenty učit tohle? \\
\note{Při formulování cílů ti pomůžou slovesa na straně \pageref{bloom}.}
\vspace*{1cm}

Jak poznám, že se cíle podařilo dosáhnout? \\
\note{Existuje test, kterým je možné zjistit, zda jsi cíle dosáhl(a)?}

\hspace*{-1cm}
\rule{\rulelength}{0.4pt}

\marginsection{90}{-2.5}{po výuce}

\smileys

\goodbadpoints

\note{Sem napiš otázku, na kterou si chceš po výuce odpovědět.}\\
Vlastní otázka:

\newpage
\marginsection{-90}{1}{volné poznámky}
\vspace*{-2em}
\note{Podařilo se mi naplnit cíl hodiny?\\
Co je třeba do budoucna na téhle hodině změnit?\\
Co je problematické na úkolech, které jsem zadal?\\
(Další tipy na otázky najdeš na straně \pageref{indikatory}.)}

% ======= WEEK 7 =======
\commonweek

% ======= WEEK 8 =======
\commonweek

% ======= WEEK 9 =======
\commonweek

% ======= WEEK 10 =======
\commonweek

% ======= WEEK 11 =======
\chapter{}
\vspace*{-2em}

\marginsection{90}{-1}{před výukou}

Příprava:\hspace{1cm} hodin(y)

Jaký je cíl této hodiny? Proč chci studenty učit tohle?
\vspace*{1cm}

Jak poznám, že se cíle podařilo dosáhnout?

\hspace*{-1cm}
\rule{\rulelength}{0.4pt}

\marginsection{90}{-3}{po výuce}

\smileys

\goodbadpoints

Blíží se konec semestru -- dostávají studenti to, co potřebují pro praxi/ke zkoušce?\\
\note{Jestli ne, jaké změny můžu v následujících týdnech provést?}

\newpage
\marginsection{-90}{1}{volné poznámky}

% ======= WEEK 12 =======

\chapter{}
\vspace*{-2em}

\marginsection{90}{-1}{před výukou}

Příprava:\hspace{1cm} hodin(y)

Jaký je cíl této hodiny? Proč chci studenty učit tohle?
\vspace*{1cm}

Jak poznám, že se cíle podařilo dosáhnout?

\hspace*{-1cm}
\rule{\rulelength}{0.4pt}

\marginsection{90}{-3}{po výuce}

\smileys

\goodbadpoints

Jaký je širší kontext toho, co učím?\\
\note{Kde v mojí výuce se dotýkám jiné výuky, jiných oblastí vědění?}

\newpage
\marginsection{-90}{1}{volné poznámky}
\vspace*{-2em}
\note{Jak získám od studentů zpětnou vazbu na ty aspekty mé vyúky, které mě zajímají?\\
Jak to dělají ostatní vyučující?\\
A jak jiné předměty?}

% ======= WEEK 13 =======

\chapter{}
\vspace*{-2em}

\marginsection{90}{-1}{před výukou}

Příprava:\hspace{1cm} hodin(y)

Jaký je cíl této hodiny? Proč chci studenty učit tohle?
\vspace*{1cm}

Jak poznám, že se cíle podařilo dosáhnout?

\hspace*{-1cm}
\rule{\rulelength}{0.4pt}

\marginsection{90}{-3}{po výuce}

\smileys

\goodbadpoints

Co mi dal tento semestr výuky?\\
\note{Co jsem si uvědomil(a)? Co jsem zlepšil(a)?}

\newpage
\marginsection{-90}{1}{volné poznámky}
\vspace*{-2em}
\note{Za co jsou studenti v předmětu hodnoceni?\\
Odpovídá to, co zkouška testuje, tomu, co já učím?\\
Pokud mají studenti zájem se dál samostatně rozvíjet v dané oblasti, jsou na to po tomto kurzu připraveni?}
