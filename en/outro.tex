% ===== OUTRO (INDICATORS, BLOOM, TOOLS) =====

\chapter*{Indicators}
\label{indicators}
\vspace{-0.5em}
\note{An aspect of teaching that can be observed directly.}

\section*{Quantitative indicators}

Time management\\
\note{How much longer/shorter was the lesson compared to the plan?\\
How long did I talk? How long were students actively engaged?\\
What was the mean time I spent with a single student?}

Student interaction\\
\note{How many questions have I asked? Were they open- or close-ended?\\
How many questions did I get? How many was I able to answer?\\
How many students did I (verbally) praise/reward for good work?}

Work on exercises\\
\note{How many exercises did students solve during the lecture?\\
How many students got lost in the tasks? How many were bored?}

Attandence\\
\note{How many students attended the lecture? How many were late?\\
How many students left during the lecture?}

\section*{Qualitative indicators}

Feelings and subjective satisfaction\\
\note{What feelings did I have during the lecture?\\
How did the students behave? Were they attentive/active?\\
Am I a good role model for the students?}

Lecture structure\\
\note{Did the studets know, what to do, how to do it and why?\\
Did I follow the planned schedule? If not, why?}

Lecture content\\
\note{Is my teaching diverse enough (task types, tools, \dots)?\\
Does my contents match the learning goals I want to reach?}

\newpage

\chapter*{Bloom's taxonomy}
\label{bloom}
\vspace{-0.5em}
\note{Hierarchy of cognitive educational objectives, B.\ Bloom, 1956}
\vspace{-0.3em}

\begin{enumerate}[leftmargin=*]
\item \textbf{Remember (\textit{knowledge})}\\
\note{facts and terminology, classification and categorization thereof}\\
{\small define, list, repeat, describe, identify, reproduce, recognize}

\item \textbf{Understand (\textit{comprehension})}\\
\note{reformulation, simple interpretation and extrapolation}\\
{\small rewrite, extend, explain, paraphrase, summarize, give example, illustrate on an example}

\item \textbf{Apply (\textit{usage})}\\
\note{applying the method in the right situation, abstracting and generalizing}\\
{\small carry out, apply, manipulate, demonstrate, implement, solve a model problem}

\item \textbf{Analyze (\textit{decomposition})}\\
\note{decomposition into basic blocks, relations and interactions between them}\\
{\small discuss, \enquote{break} into smaller parts, compare and contrast, design/select a solution, deconsruct, interconnect}

\item \textbf{Evaluate (\textit{judgements})}\\
\note{assessment based on set criteria and standards}\\
{\small evaluate, conclude, test, assess, ctiticize, justify}

\item \textbf{Create (\textit{synthesis})}\\
\note{creating a new product, reorganization into a new structure}\\
{\small generate, modify, rearrange, invent, design, budovat, compose}
\end{enumerate}

\note{Level boundaries are not strict. The list of actions can help you describe the learning goals/objectvives and knowledge/skills the students should gain. It can indicate how diverse your teaching is.}

\chapter*{Useful teaching tools}
\vspace{-0.5em}
\note{A handful of tools and concepts useful for teaching}

\section*{The broader context of teaching}

\begin{itemize}
\item Používání jmen -- \note{jmenovky, tahák, ptaní se}
\item Uspořádání prostoru -- \note{rovnost, tvar usazení, pozice učitele}
\item Verbalizace očekávání -- \note{očekávají obě strany to stejné?}
\item Precedenty -- \note{historie, co se stalo a může se opakovat}
\item Zakázky -- \note{co studenti chtějí dostat?}
\item Hospitace -- \note{jak to dělají jiní? Jak to dělám já?}
\item Žádost o zpětnou vazbu -- \note{je to skutečně tak, jak si myslím?}
\item Bloomova taxonomie -- \note{verbalizace cílů hodiny/výuky}
\item Rubrika -- \note{self-assessment, sledování pokroku}
\end{itemize}

\section*{Lecture structure}

\begin{itemize}
\item Situování -- \note{shrnutí kde jsme a kam jdeme}
\item Označení přechodů -- \note{explicitní přechod do dalšího bloku}
\item Veřejný checklist -- \note{viditelná struktura hodiny}
\item Check-in -- \note{otevření, zjištění stavu studentů}
\item Check-out -- \note{uzavření, výzva k reflexi}
\item Tracking -- \note{sledování/uvědomování si vývoje diskuze/hodiny}
\item Parkoviště otázek -- \note{odkládání větších otázek na později}
\end{itemize}

\section*{Assigning tasks/exercises}

\begin{itemize}
\item Otázky ke skupině -- \note{srozumitelnost, podmínka + akce}
\item Multiple choice questions -- \note{uvěřitelnost možností}
\item Hlasování -- \note{pozitivní, negativní, rozdělování bodů, Kahoot}
\item Jasnost zadání -- \note{začátek, proces, výsledek, hodnocení, trvání}
\item Praktická ukázka -- \note{aby se museli i dívat, nejen poslouchat}
\item Rozcvička -- \note{fyzické/myšlenkové probuzení, opakování}
\item Nastavování prahu -- \note{jak moc snadné je to udělat?}
\item Chunking -- \note{dělení na malý počet stravitelných části}
\item Externí motivace -- \note{body navíc, bonbóny, ...}
\end{itemize}

\section*{Creating new tasks/exercises}

\begin{itemize}
\item Schody -- \note{postupnost malých, ztěžujících se úkolů}
\item Netradiční zadání -- \note{např.\ programování na mnoho způsobů}
\item Argumentace -- \note{zdůvodnění svého názoru, hodnocení jiných}
\item Figurky -- \note{fyzické objekty a manipulace s nimi}
\item Antiproblém -- \note{řešení opačného problému}
\item Myšlenkový oblak -- \note{sbírání asociací, diskuze, souvislosti}
\item Konceptová mapa -- \note{vizualizace pojmů/znalostí a jejich vztahů}
\end{itemize}

\section*{Uvádění aktivit}

\begin{itemize}
\item Subgrouping -- \note{dělení skupiny na menší části}
\item Harvest/sklizeň -- \note{sdílení názorů, zesílení důležitých signálů}
\item Think-pair-share -- \note{nejprve sami, diskuze v páru, sklízení}
\item Peer review -- \note{studenti si navzájem hodnotí práci}
\item Peer tutoring -- \note{studenti se učí navzájem}
\item Neupozornění na chybu -- \note{studenti si ji najdou sami}
\item Obrácená hodina -- \note{samostudium teorie, společné cvičení}
\end{itemize}

\chapter*{Your own comments}
\note{The most relavant feedback from your colleagues for example.}

\chapter*{Your own comments}
\note{What should I concentrate on when teaching?}
