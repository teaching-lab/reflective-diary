% ===== INTRODUCTION =====

\chapter*{Jak se z člověka stane dobrý učitel?}

\vspace*{1em}
Položme si namísto statické otázky dynamickou:

\vspace*{1em}
\textit{\large \enquote{Co můžeš dělat pro to, aby ses jako učitel zlepšoval?}}

\vspace*{1em}
Ukazuje se, že jsou nutné, a nejspíš i dostačující, tři věci\punct{.}\footnotemark
\footnotetext{Jsou i další věci, které mohou pomoci (například expertní zpětná vazba), ale ty nejsou dobře škálovatelné.}
\begin{enumerate}
\item \textbf{Pravidelně učit.} \\Potřebuješ mít vlastní zkušenost, nejlépe často a pravidelně opakovanou.
\item \textbf{Reflektovat svoje učení.} \\Všímej si a přemýšlej nad tím, co fungovalo dobře a co příště zkusíš jinak.
\item \textbf{Pozorovat, jak učí ostatní.} \\Přemýšlej o tom, jak to dělají, co jim funguje, co ne a co z toho můžeš použít.
\end{enumerate}

Tento deníček ti pomůže reflektovat způsob, jakým učíš. Také v~něm nalezneš informace o tom, čeho je užitečné si všímat, jaké otázky si klást a nad čím přemýšlet. Není to však učebnice dobrého učitele -- je to pouze průvodce na cestě, po které už jdeš.

Vhodným doplňkem k vlastní reflexi výuky jsou diskuze s ostatními učiteli o tom, jak učí. Pokud se chceš setkávat s dalšími cvičícími pravidelně a navíc se dozvědět více o tom, co funguje a co ne, zapiš si předmět \textit{DUCIT Praktikum vedení cvičení} na Fakultě informatiky MU.

\newpage
\section*{Jak reflektivní deníček používat?}

Nejlépe pravidelně.

Při přípravě i po každé své odvedené výuce si zapiš svoje myšlenky. Na třináct týdnů v semestru máš nachystaných třináct dvoustránek, a na každé pár nabídnutých otázek. Deníček je malý mimo jiné i proto, aby ti psaní nezabralo moc času.

Za stránkami pro jednotlivé týdny najdeš ještě rubriku učitelských schopností, seznam indikátorů, které je možné na hodině sledovat, připomenutí některých užitečných nástrojů k výuce a na závěr prostor na své vlastní poznámky.

Někde záměrně používáme anglické termíny (jako například \textit{self-assessment}), protože vhodný český překlad buď neexistuje, nebo má konotace, které nechceme.

\section*{Jaké jsou cíle používání deníku?}

Chtěli bychom, aby ti práce s deníkem:
\begin{enumerate}[topsep=0pt]
\item Připomínala zamyslet se nad svou hodinou (reflexe).
\item Dávala prostor pro snadné psaní si poznámek pro příště.
\item Pomáhala uvědomit si, co všechno patří k tvojí roli učitele.
\item Umožňovala sledovat svůj vlastní pokrok.
\end{enumerate}

\section*{Máš nápady na zlepšení?}

Budeme rádi za jakékoliv nápady, zkušenosti či poznámky k deníku. Ozvi se prosím emailem Martinovi na \textit{mukrop@mail.muni.cz}.
