% ======= TODO LIST =======

% Printing: https://tex.stackexchange.com/questions/26628/printing-a6-on-a4-paper

% ======= STANDARD DOCUMENT SETTINGS AND PACKAGES ========

\documentclass[twoside,openany,10pt]{book}
\usepackage[czech]{babel}			% Language
\usepackage[utf8]{inputenc}			% Encoding of characters in this .tex file
\usepackage{cmap}					% Make PDF file searchable and copyable (ASCII characters)
\usepackage{lmodern}				% Make PDF file searchable and copyable (Accented characters)
\usepackage[T1]{fontenc}			% Hyphenate accented words
\usepackage[a6paper, top=0.75cm, bottom=1.5cm, inner=0.75cm, outer=0.75cm]{geometry}		% Paper size and margins
\usepackage{marginnote}				% enable margin notes
\usepackage{rotating}				% Rotated environments (for margin headers)
\usepackage[protrusion]{microtype}	% Better typeset results
\usepackage{enumitem}				% Nicer enumeration lists
\usepackage{ragged2e}				% Add \justify command
\usepackage{titlesec}				% Customize chapters
\usepackage[usenames,dvipsnames,svgnames,table]{xcolor}		% Custom colors
\usepackage{tikzsymbols}			% Smileys
%\usepackage[colorlinks]{hyperref}	% Hyperlinks inside the document for table of contents

% ======= CUSTOM DOCUMENT SETTINGS ========

% Change \chapter and \section command display and spacings
% Source: https://tex.stackexchange.com/questions/63390/how-to-decrease-spacing-before-chapter-title
\titleformat{\chapter}[display]{\normalfont\Large\bfseries}{Týden \thechapter}{0cm}{\Large}
\titlespacing*{\chapter}{0em}{-2em}{0em}
\titleformat{\section}[display]{\normalfont\large\bfseries}{\sectionname}{}{\large}
\titlespacing*{\section}{0em}{0em}{0.5em}

% Remove headers and position page number in the bottom mid
\pagestyle{plain}

% Smaller spaces between enumeration lists
\setitemize{itemsep=0em}

% Disable automatic indentation, enable paragraph skip
\setlength{\parindent}{0em}
\setlength{\parskip}{0.5em}

% Custom environments
\newcommand{\note}[1]{\textcolor{gray}{\small\itshape #1}}

\newcommand{\marginsection}[3]{\marginpar{\raisebox{#2\height}{
\begin{turn}{#1}
%\fontfamily{phv}\selectfont 
\bfseries \color{gray} \large
#3\end{turn}}}}

\newcommand{\rubricpage}[8]{
\newpage
\begin{tabular}{@{}>{\small}C{0.48\textwidth}>{\small}C{0.48\textwidth}@{}l@{}}
\normalsize \bfseries #1 & \normalsize \bfseries #5 & \\[1em] \hline \\[-1.2em]
\justify \note{Not aware:} #2 & \justify \note{Not aware:} #6 & \\[2em]
$\bullet$ & $\bullet$ & \\
$\bullet$ & $\bullet$ & \\
\justify \note{Novice skill:} #3 & \justify \note{Novice skill:} #7 & \\[3em]
$\bullet$ & $\bullet$ & \\
$\bullet$ & $\bullet$ & \\
$\bullet$ & $\bullet$ & \\
$\bullet$ & $\bullet$ & \\
$\bullet$ & $\bullet$ & \\
\justify \note{Mastery:} #4 & \justify \note{Mastery:} #8 &
\end{tabular}
}

\newcommand{\commonweek}{
\chapter{}
\vspace*{-2em}

\marginsection{90}{-1}{před cvičením}

Čas strávený přípravou na cvičení: \hspace{1cm} hodin.

Jaký je cíl tohoto cvičení? Proč to učíme?
\vspace*{1cm}

Zamyšlení: Jak poznáš, že se toho podařilo dosáhnout?

\hspace*{-1cm}
\rule{\rulelength}{0.4pt}

\marginsection{90}{-1.2}{po cvičení}

Jak jsem spokojen se cvičením?
\raisebox{-0.3em}{\quad\Changey[2]{-2}\,\,\Changey[2]{-1}\,\,\Changey[2]{0}\,\,\Changey[2]{1}\,\,\Changey[2]{2}}

Co se mi podařilo?
\begin{enumerate}
\item 
\item
\end{enumerate}

Co mohlo být líp?
\begin{enumerate}
\item 
\item
\end{enumerate}

Vlastní otázka:

\newpage
\marginsection{-90}{1}{volné poznámky}
}

% Rule length including the margin
\newlength{\rulelength}
\setlength{\rulelength}{\dimexpr(\textwidth+0.5cm)\relax}

% New table column types with flexible width
%\newcolumntype{C}{>{\centering\arraybackslash}X}
\newcolumntype{C}[1]{>{\centering\arraybackslash}m{#1}}

% Rotated table cell
\newcommand{\rotatedCell}[2][l]{\rotatebox{90}{\renewcommand{\arraystretch}{0.75}\begin{tabular}[#1]{@{}l}#2\end{tabular}}}

% ======= TITLE PAGE =======

\title{Reflektivní deník učitele}
\author{Kolektiv vyučujících Praktika vedení cvičení}
\date{\today}

% ======= DOCUMENT START =======

\begin{document}
\begin{titlepage}
	\centering
    \vspace*{1cm}
	\includegraphics[width=0.5\textwidth]{dennik}\par
	\vspace{0.5cm}
	{\huge\bfseries Reflektivní\\ deník učitele\par}
	\vfill
	\raggedright
    \Large
	Majitel:\par
    \vspace{0.2cm}
    Předmět:\par
\end{titlepage}

% ======= INTRODUCTION =======

\chapter*{Úvod}

Lorem ipsum dolor sit amet, consectetur adipiscing elit, sed do eiusmod tempor incididunt ut labore et dolore magna aliqua. Ut enim ad minim veniam, quis nostrud exercitation ullamco laboris nisi ut aliquip ex ea commodo consequat. Duis aute irure dolor in reprehenderit in voluptate velit esse cillum dolore eu fugiat nulla pariatur. Excepteur sint occaecat cupidatat non proident, sunt in culpa qui officia deserunt mollit anim id est laborum.

\newpage
Druha strana uvodu

\newgeometry{top=0.75cm, bottom=1.5cm, inner=0.75cm, outer=1.2cm,
			heightrounded, marginparwidth=0.5cm, marginparsep=0.3cm}

% ======= WEEK 1 =======

\chapter{}
\vspace*{-2em}

\marginsection{90}{-2.1}{před cvičením}

Čas strávený přípravou na cvičení: \hspace{1cm} hodin.\\
\note{Příprava slidů, úkolů, pokynů, \dots}

Jaký je cíl tohoto cvičení? Proč to učíme? \\
\note{Mohou ti pomoct slovesa na straně \pageref{bloom}.}
\vspace*{1cm}

Zamyšlení: Jak poznáš, že se toho podařilo dosáhnout? \\
\note{Myšlenka \uv{test-driven-teaching}.}

Jaké precedenty chci ve cvičení nastavit? \\
\note{Pokládání otázek, meškání, formálnost, očekávání, ...}
\vspace*{1cm}

\hspace*{-1cm}
\rule{\rulelength}{0.4pt}

\marginsection{90}{-1.2}{po cvičení}

Jak jsem spokojen se cvičením?
\raisebox{-0.3em}{\quad\Changey[2]{-2}\,\,\Changey[2]{-1}\,\,\Changey[2]{0}\,\,\Changey[2]{1}\,\,\Changey[2]{2}} \\
\note{Jaký mám já pocit? Co vidím na studentech?}

Co se mi podařilo?
\begin{enumerate}
\item 
\item
\end{enumerate}

Co mohlo být líp?
\begin{enumerate}
\item 
\item
\end{enumerate}

\newpage
\marginsection{-90}{1}{volné poznámky}
\vspace*{-2em}
\note{Kolik otázek do publika jsem položil(a)?\\
Kolik otázek položili studenti mně? Je to málo/dost/příliš?\\
Podařilo se mi naplnit cíl cvičení?\\
Co je třeba do budoucna na téhle hodině změnit?\\
(Další tipy na otázky najdeš na straně \pageref{indikatory}.)}

% ======= WEEK 2 =======

\chapter{}
\vspace*{-2em}

\marginsection{90}{-2.1}{před cvičením}

Čas strávený přípravou na cvičení: \hspace{1cm} hodin.\\
\note{Příprava slidů, úkolů, pokynů, \dots}

Jaký je cíl tohoto cvičení? Proč to učíme? \\
\note{Mohou ti pomoct slovesa na straně \pageref{bloom}.}
\vspace*{1cm}

Zamyšlení: Jak poznáš, že se toho podařilo dosáhnout? \\
\note{Myšlenka \uv{test-driven-teaching}.}

Jaké precedenty chci ve cvičení nastavit? \\
\note{Pokládání otázek, meškání, formálnost, očekávání, ...}
\vspace*{1cm}

\hspace*{-1cm}
\rule{\rulelength}{0.4pt}

\marginsection{90}{-1.2}{po cvičení}

Jak jsem spokojen se cvičením?
\raisebox{-0.3em}{\quad\Changey[2]{-2}\,\,\Changey[2]{-1}\,\,\Changey[2]{0}\,\,\Changey[2]{1}\,\,\Changey[2]{2}} \\
\note{Jaký mám já pocit? Co vidím na studentech?}

Co se mi podařilo?
\begin{enumerate}
\item 
\item
\end{enumerate}

Co mohlo být líp?
\begin{enumerate}
\item 
\item
\end{enumerate}

\newpage
\marginsection{-90}{1}{volné poznámky}
\vspace*{-2em}
\note{Kolik otázek do publika jsem položil(a)?\\
Kolik otázek položili studenti mně? Je to málo/dost/příliš?\\
Podařilo se mi naplnit cíl cvičení?\\
Co je třeba do budoucna na téhle hodině změnit?\\
(Další tipy na otázky najdeš na straně \pageref{indikatory}.)}

% ======= WEEK 3 =======

\chapter{}
\vspace*{-2em}

\marginsection{90}{-1}{před cvičením}

Čas strávený přípravou na cvičení: \hspace{1cm} hodin.\\
\note{Příprava slidů, úkolů, pokynů, \dots}

Jaký je cíl tohoto cvičení? Proč to učíme? \\
\note{Mohou ti pomoct slovesa na straně \pageref{bloom}.}
\vspace*{1cm}

Zamyšlení: Jak poznáš, že se toho podařilo dosáhnout? \\
\note{Myšlenka \uv{test-driven-teaching}.}

\hspace*{-1cm}
\rule{\rulelength}{0.4pt}

\marginsection{90}{-1.2}{po cvičení}

Jak jsem spokojen se cvičením?
\raisebox{-0.3em}{\quad\Changey[2]{-2}\,\,\Changey[2]{-1}\,\,\Changey[2]{0}\,\,\Changey[2]{1}\,\,\Changey[2]{2}} \\
\note{Jaký mám já pocit? Co vidím na studentech?}

Co se mi podařilo?
\begin{enumerate}
\item 
\item
\end{enumerate}

Co mohlo být líp?
\begin{enumerate}
\item 
\item
\end{enumerate}

\note{Sem napiš otázku, na kterou si chceš po cvičení odpovědět.}\\
Vlastní otázka:

\newpage
\marginsection{-90}{1}{volné poznámky}
\vspace*{-2em}
\note{Kolik otázek do publika jsem položil(a)?\\
Kolik otázek položili studenti mně? Je to málo/dost/příliš?\\
Podařilo se mi naplnit cíl cvičení?\\
Co je třeba do budoucna na téhle hodině změnit?\\
(Další tipy na otázky najdeš na straně \pageref{indikatory}.)}

% ======= WEEK 4 =======

\chapter{}
\vspace*{-2em}

\marginsection{90}{-1}{před cvičením}

Čas strávený přípravou na cvičení: \hspace{1cm} hodin.\\
\note{Příprava slidů, úkolů, pokynů, \dots}

Jaký je cíl tohoto cvičení? Proč to učíme? \\
\note{Mohou ti pomoct slovesa na straně \pageref{bloom}.}
\vspace*{1cm}

Zamyšlení: Jak poznáš, že se toho podařilo dosáhnout? \\
\note{Myšlenka \uv{test-driven-teaching}.}

\hspace*{-1cm}
\rule{\rulelength}{0.4pt}

\marginsection{90}{-1.2}{po cvičení}

Jak jsem spokojen se cvičením?
\raisebox{-0.3em}{\quad\Changey[2]{-2}\,\,\Changey[2]{-1}\,\,\Changey[2]{0}\,\,\Changey[2]{1}\,\,\Changey[2]{2}} \\
\note{Jaký mám já pocit? Co vidím na studentech?}

Co se mi podařilo?
\begin{enumerate}
\item 
\item
\end{enumerate}

Co mohlo být líp?
\begin{enumerate}
\item 
\item
\end{enumerate}

\note{Sem napiš otázku, na kterou si chceš po cvičení odpovědět.}\\
Vlastní otázka:

\newpage
\marginsection{-90}{1}{volné poznámky}
\vspace*{-2em}
\note{Kolik otázek do publika jsem položil(a)?\\
Kolik otázek položili studenti mně? Je to málo/dost/příliš?\\
Podařilo se mi naplnit cíl cvičení?\\
Co je třeba do budoucna na téhle hodině změnit?\\
(Další tipy na otázky najdeš na straně \pageref{indikatory}.)}

% ======= WEEK 5 =======
\commonweek

% ======= WEEK 6 =======
\commonweek

% ======= WEEK 7 =======
\commonweek

% ======= WEEK 8 =======
\commonweek

% ======= WEEK 9 =======
\commonweek

% ======= WEEK 10 =======
\commonweek

% ======= WEEK 11 =======
\commonweek

% ======= WEEK 12 =======

\chapter{}
\vspace*{-2em}

\marginsection{90}{-1}{před cvičením}

Čas strávený přípravou na cvičení: \hspace{1cm} hodin.

Jaký je cíl tohoto cvičení? Proč to učíme?
\vspace*{1cm}

Zamyšlení: Jak poznáš, že se toho podařilo dosáhnout?

\hspace*{-1cm}
\rule{\rulelength}{0.4pt}

\marginsection{90}{-1.2}{po cvičení}

Jak jsem spokojen se cvičením?
\raisebox{-0.3em}{\quad\Changey[2]{-2}\,\,\Changey[2]{-1}\,\,\Changey[2]{0}\,\,\Changey[2]{1}\,\,\Changey[2]{2}}

Co se mi podařilo?
\begin{enumerate}
\item 
\item
\end{enumerate}

Co mohlo být líp?
\begin{enumerate}
\item 
\item
\end{enumerate}

Otázka 12:\\
\note{Notes.}

\newpage
\marginsection{-90}{1}{volné poznámky}
\vspace*{-2em}
\note{Kolik otázek do publika jsem položil(a)?\\
Kolik otázek položili studenti mně? Je to málo/dost/příliš?\\
Podařilo se mi naplnit cíl cvičení?\\
Co je třeba do budoucna na téhle hodině změnit?\\
(Další tipy na otázky najdeš na straně \pageref{indikatory}.)}

% ======= WEEK 13 =======

\chapter{}
\vspace*{-2em}

\marginsection{90}{-1}{před cvičením}

Čas strávený přípravou na cvičení: \hspace{1cm} hodin.

Jaký je cíl tohoto cvičení? Proč to učíme?
\vspace*{1cm}

Zamyšlení: Jak poznáš, že se toho podařilo dosáhnout?

\hspace*{-1cm}
\rule{\rulelength}{0.4pt}

\marginsection{90}{-1.2}{po cvičení}

Jak jsem spokojen se cvičením?
\raisebox{-0.3em}{\quad\Changey[2]{-2}\,\,\Changey[2]{-1}\,\,\Changey[2]{0}\,\,\Changey[2]{1}\,\,\Changey[2]{2}}

Co se mi podařilo?
\begin{enumerate}
\item 
\item
\end{enumerate}

Co mohlo být líp?
\begin{enumerate}
\item 
\item
\end{enumerate}

Otázka 13:\\
\note{Notes.}

\newpage
\marginsection{-90}{1}{volné poznámky}
\vspace*{-2em}
\note{Kolik otázek do publika jsem položil(a)?\\
Kolik otázek položili studenti mně? Je to málo/dost/příliš?\\
Podařilo se mi naplnit cíl cvičení?\\
Co je třeba do budoucna na téhle hodině změnit?\\
(Další tipy na otázky najdeš na straně \pageref{indikatory}.)}

% ======= APPENDICES =======

\restoregeometry
\chapter*{Rubrika}

Uvodny text k rubrike -- co to je, kedy a ako sa to ma pouzit.
\begin{itemize}
\item Co to je?
\item Ako to pouzit? (2-3x za rok si vyplnit vsetky skaly, kde sa aktualne nechadzam? Vybrat si 1-2 oblasti, kde sa chcem zlepsit, vymysliet konkretne kroky, najst si indikatory na sledovanie stavu, napisat si ich do stran k tyzdnom?)
\item Ine pouzitie: Moznost precitat si Mastery popisy a uvedomit si, co vsetko ucitel robi/mal by robit.
\end{itemize}

\newpage
\rubricpage{Vědomá pozornost a rozhodování, \textit{tracking}}
{Nedokážu si při výuce ani po ní vybavit, co se děje a jaký mám záměr. Při výuce se často cítím ztracený.}
{Někdy se mi daří uvědomovat si, jaký mám právě záměr, co právě dělám a jaký to bude mít efekt. Většinou ale nemám při výuce vědomou pozornost.}
{Během výuky jsem si téměř vždy vědom toho, co se se mnou i se skupinou děje, co dělám a jaký to bude mít efekt, kam směřuji. Jsem schopen si vybavit, jak jsme se dostali do tam, kde právě jsme.}
{Interakce se skupinou, kladení otázek skupině}
{Se skupinou neinteraguji. Neptám se, asi bych nedostal odpověď. Nevím, jak studenty dobře zapojit.}
{Vím, že lze interagovat se skupinou a znám nástroje (chápu, jak by to šlo). Nejsem ale schopen je často používat. Někdy  se skupiny ptám, ale nedostávám odpověď.}
{Se skupinou často a efektivně interaguji, ptám se způsobem, který studenty aktivizuje a zapojuje. Situace, kdy nedostávám ze skupiny odpověď dokážu obratně řešit (např. přeformulováním otázky).}

\rubricpage{Strukturování výuky}
{O strukturování výuky nijak nepřemýšlím.}
{Chápu smysl přehledného strukturování své hodiny a snažím se o to. Často se ale zamotám, ztratím nit, nebo řeším mnoho naráz a studenti se pak ztrácí nebo odpojují.}
{Moje cvičení mají jasnou strukturu. Studenti ví co se právě děje, co bude následovat a chápou návaznosti. Mezi jednotlivými bloky vědomě dělám zřetelné přechody.}
{Vlastní pocit z výuky}
{Svou výuku nijak nereflektuji.}
{Ve výuce si často moc nevěřím, vyžaduje to hodně energie, nebo se cítím napjatý. Mám strach z toho, na co se studenti zeptají.}
{Ve výuce se většinou cítím uvolněně a sebevědomě, mám svůj styl a baví mě to.}

\rubricpage{Formativní zpětná vazba studentům}
{O dávání zpětné vazby studentům nijak nepřemýšlím.}
{Snažím se studentům zpětnou vazbu dávat. Myslím si však, že jí není dost, nebo to nedělám efektivně, nebo to studenti nevnímají jako podporu a projev respektu.}
{Se studenty ve výuce interaguji tak, že dostávají průběžně formativní zpětnou vazbu. Chápou tedy, co jim jde, kde dělají chyby a jak se mohou zlepšovat. Zároveň se moji studenti cítí být respektováni a zpětné vazby se nebojí.}
{Jasné zadávání úloh}
{O zadávání úloh nijak zvlášť nepřemýšlím.}
{Stává se mi, že zadám úlohu a studenti neví, co dělat, jak začít nebo k čemu mají dojít (co má být výsledkem).}
{Když zadávám úlohu nebo aktivitu, mají studenti jasno v tom, co mají dělat a neřeší tak zbytečně věci, které nemám záměr procvičovat.}

\rubricpage{Design výuky a variabilita aktivit}
{Učím tak, jak mi řekli nebo kopíruji výuku, kterou jsem viděl. Nepřemýšlím o jiných variantách.}
{Jsem si vědom toho, že existuje mnoho typů aktivit, které lze při výuce použít. Neznám jich ale dostatek, nejsem je schopen efektivně zadávat, nebo nemám jasno v tom proč je vybírat.}
{Znám dostatek typů aktivit, svou výuku skládám tak, aby byla dostatečně pestrá. Vybrané aktivity efektivně učí/procvičují to, co mám záměr procvičovat. Vybrané aktivity zároveň studenty efektivně zapojují a přispívají k jejich motivaci.}
{Širší kontext výuky a konkrétního cvičení}
{O širším kontextu cvičení nepřemýšlím.}
{Uvědomuji si, že nemám pro sebe pojmenované znalosti a dovednosti, které u studentů rozvíjím. Nevím, jak jejich pokrok sledovat. Nevím, v jakém kontextu to studenti využijí.}
{Mám jasnou představu o tom, k čemu studenty vedu, jaké dovednosti rozvíjím, jaké znalosti jim chci předat. Vím, proč tyto dovednosti rozvíjím a v jakém kontextu je studenti v budoucnu použijí. Vím, jak jejich pokrok sledovat.}

\rubricpage{Jasné vysvětlování}
{Svoje vysvětlování nijak nereflektuji.}
{Když vysvětluji, běžně se mi stává, že si nejsem jistý, zda vysvětluji dobře a zda to studentům pomáhá při pochopení.}
{Když vysvětluji teorii, demonstruji řešení nebo opravuji chybný postup. Dělám to efektivně a dokážu se dobře vžít do toho, jak to vidí student. Nestává se mi, že by studenti moje vysvětlení nechápali, nebo že bych vysvětloval něco, na co se neptali.}
{Nastavení prostředí, systémy ve výuce}
{O nastavení atmosféry nepřemýšlím, systémy ve své výuce nevnímám.}
{Přemýšlím o nastavení pravidel i atmosféry. Systémy ve výuce (např. bodování) přebírám od ostatních. Nemám ale jasno v efektech, nebo neznám způsoby, jak bych je mohl upravit.}
{Umím ve výuce vytvořit prostředí, které podporuje efektivní učení. U systémů, které používám (např. bodování, bonbóny, zahajovací rituály) chápu efekt. Systémy nepřebírám slepě, chápu jejich účel a uzpůsobuji podle svých potřeb.}

\rubricpage{Improvizace, přizpůsobování}
{V průběhu své výuky na vzniklé situace nijak vědomě nereaguji.}
{Uvědomuji si chvíle, kdy by mohlo být zajímavé nebo užitečné dělat něco jiného, než jsem měl v plánu. Většinou ale nedokážu v daný moment vhodně zareagovat.}
{Jsem schopen svoji výuku průběžně přizpůsobovat tomu, co se právě děje ve skupině a co studenti potřebují. Mám k tomu dostatek nástrojů a dokážu je efektivně použít.}
{Individuální interakce se studenty}
{O konzultacích nijak vědomě nepřemýšlím.}
{Stává se mi, že si nevím rady při individuální interakci se studentem (např. u tabule, konzultace). Interakce neprobíhá efektivně nebo se student cítí zastrašen.}
{Při interakcích s jednotlivci (např. u tabule, konzultace) efektivně využívám čas. Studenti se mnou konzultují rádi, cítí ze mě respekt a podporu.}

\rubricpage{Management skupiny, \textit{subgrouping}}
{Přijde mi, že dělit skupinu je zbytečnost, o~práci v~menších skupinkách nepřemýšlím.}
{Uvědomuji si možnosti práce ve dvojicích či menších skupinách. Tuším, že bych toho mohl lépe využívat, hledám jak.}
{Mám jasno v tom, kdy chci pracovat se skupinou jako celkem, kdy s jednotlivci a kdy se skupinkami. Rozdělení do menších skupin efektivně používám. Ve vhodných případech zadávám interakce mezi skupinami.}
{Čtení skupiny}
{Skupinu ve svém cvičení nesleduji, pozornost věnuji pouze obsahu.}
{Uvědomuji si, že mi skupina vysílá signály a že by bylo dobré jim rozumět a využít je pro efektivní vedení cvičení. Ve výuce to ale dokážu jen výjimečně.}
{Dokážu dobře odhadnout naladění skupiny. Mám jasno v tom, co se ve skupině studentů děje (např. únava, rezignace, zájem).}

\chapter*{Indikátory}
\label{indikatory}

\section*{Kvalitativní}
\begin{itemize}
\item Jaké pocity jsem měl na cvičení?
\item Poslouchali mě studenti?
\item Podařilo se mi dodržet časový plán?
\item Byli studenti aktivní?
\end{itemize}

\section*{Kvantitativní}
\begin{itemize}
\item Kolik otázek jsem položil?
\item Kolik otázek jsem dostal?
\item Kolik času zůstalo navíc?
\item O kolik minut jsem prodlužoval?
\item Kolik úloh se vyřešilo?
\item Kolik úkolů dostali studenti na doma?
\item Jak dlouho jsem měl monolog?
\item Kolik studentů přišlo na cvičení?
\item Kolik studentů odešlo během cvičení?
\end{itemize}
\newpage

\chapter*{Bloomova taxonomie}
\label{bloom}

\begin{enumerate}[leftmargin=*]
\item \textbf{Vzpomenout si (\textit{remember})}\\
\note{termíny a fakta, jejich klasifikace a kategorizace}\\
{\small definovat, vyjmenovat, zopakovat, popsat, rozpoznat, přiřadit pojem ke konceptu}

\item \textbf{Pochopit (\textit{understand})}\\
\note{překlad do jiné formy, jednoduchá interpretace, extrapolace}\\
{\small interpretovat, reprezentovat, vysvětlit, parafrázovat, sumarizovat, uvést příklad, ilustrovat na příkladě, porovnat}

\item \textbf{Použít (\textit{apply})}\\
\note{užití postupu ve správné situaci, použití abstrakcí a zobecnění}\\
{\small vykonat, aplikovat, použít, implementovat, vyřešit uzavřený a ohraničený problém}

\item \textbf{Analyzovat (\textit{analyze})}\\
\note{rozklad na části, vztahy a interakce mezi částmi}\\
{\small diskutovat (\uv{rozbít} na části), rozlišit, zvolit, organizovat, integrovat, strukturovat}

\item \textbf{Vyhodnotit (\textit{evaluate})}\\
\note{posouzení na základě stanovených kritérií a standardů}\\
{\small zkontrolovat, detekovat, testovat, posoudit, odborně kritizovat, argumentovat}

\item \textbf{Vytvořit (\textit{create})}\\
\note{vytvoření nového celku, reorganizace do nové struktury}\\
{\small generovat, plánovat, designovat, budovat, produkovat, kreativně kombinovat prvky}
\end{enumerate}

\chapter*{Užitečné nástroje}

\section*{Základní nástroje}
\begin{enumerate}
\item Otázky -- jasné, intonačně neutrální
\item Samostatná práce -- odhad trvání
\item Domácí úkoly -- předávání a hodnocení
\end{enumerate}

\section*{Alternativní nástroje dotazování se}
\begin{enumerate}
\item Hlasování -- kdo si myslí, že \textit{výrok}, ať \textit{akce}
\item Kahoot
\end{enumerate}

\section*{Alternativní nástroje samostatné práce}
\begin{enumerate}
\item Think-Pair-Share -- sami, poté v páru
\item Anti-problém -- řešení opaku problému
\item Myšlenkový oblak -- co vás napadne, když sa řekne \textit{slovo}
\item Rozcvička -- krátké opakování
\item Skládání kartiček -- z částí sestavit celek
\item Hry -- zábavné aktivity na danou látku
\end{enumerate}

\section*{Alternativní nástroje domácích úkolů}
\begin{enumerate}
\item Bonusy za splnění něčeho navíc
\item Kontrolování úkolů studenty navzájem
\item Společná revize vybraného domácího úkolu -- proč je řešení dobré / špatné
\end{enumerate}

\section*{Základní nástroje vysvětlování}
\begin{enumerate}
\item Časový plán -- kolik má která část trvat
\item Zapojení studentů do výkladu
\item Získávání zpětné vazby -- ujištění se, že studenti pochopili správně to, co se řeklo
\item Praktická ukázka -- aby se museli i dívat, nejen poslouchat
\item Slajdy (případně checkboxy na tabuli)\\ -- vizualizace aktuálního pokroku hodiny
\item Vtip -- odlehčení atmosféry
\end{enumerate}

\chapter*{Prostor pro poznámky}
\note{Například nejdůležitejší feedback z hospitací.}

\chapter*{Prostor pro poznámky}
\note{Na co chci myslet, když učím?}

\newpage
\vspace*{\fill}
Verze: \today

Vytvořeno pro kurz \textit{DUCIT Praktikum vedení cvičení} na Fakultě informatiky MU.

Autoři: František Blahoudek?, Martin Macák, Michaela Pokludová?, Ondráš Přibyla, Valdemar Švábenský, Martin Ukrop

\end{document}
